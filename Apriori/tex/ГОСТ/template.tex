%
\documentclass[a4paper,14pt]{article}
\usepackage[utf8]{inputenc}
\RequirePackage[english,russian]{babel}
\usepackage{fullpage}
\usepackage{verbatim}
\usepackage{minitoc}
\usepackage{mathtext}
\usepackage{graphicx}
\usepackage{caption2}
\usepackage{epsfig,anysize,amssymb,verbatim,hhline,texnames,subfigure,multicol, wrapfig, amsmath}
% \usepackage{G7-32}
\renewcommand{\captionlabeldelim}{.}
\renewcommand{\baselinestretch}{1.5}
\title{Дипломная работа}
\author{Костенчук М.И.}

\date{31-03-2011}

\begin{document}

\maketitle

\tableofcontents
\newpage

\section{Введение}

Невозможно представить современную жизнь при отсутствии дорожного транспорта. Сегодня у многих жителей России есть свой личный автомобиль, тогда как отыскать человека, никогда в жизни не пользовавшегося общественным транспортом довольно сложная задача. Это безусловно удобно. Но с развитием этой отрасли, неизбежно и появление издержек. Машин становится всё больше, плотность проживания людей в результате урбанизации увеличивается. Эти тенденции приводят к увеличению загруженности транспортных систем. Встаёт вопрос о том, чтобы в автоматическом режиме обнаруживать слабые места в транспортной сети и распределять нагрузку с наиболее загруженных участков.

Для решения этих проблем и были придуманы интеллектуальные транспортные системы. ИТС — это интеллектуальная система, использующая инновационные разработки в моделировании транспортных систем и регулировании транспортных потоков, предоставляющая конечным потребителям большую информативность и безопасность, а также качественно повышающая уровень взаимодействия участников движения по сравнению с обычными транспортными системами. Особенность ИТС относительно других видов наблюдения за транспортными потоками или их моделирования заключается в том, что ИТС даёт возможность в автоматическом режиме сигнализировать о наступлении событий, ведущих к возникновению на дороге проблемных ситуаци.

ИТС включает в себя информационные и коммуникационные технологии в сфере автотранспорта (включая инфраструктуру, транспортные средства, участников системы, а также дорожно-транспортное регулирование), и имеющую наряду с этим возможность взаимодействия с другими видами транспорта.[DIRECTIVE 2010/40/EU OF THE EUROPEAN PARLIAMENT AND OF THE COUNCIL]

Одной из важнейших задач интеллектуальных транспортных систем является заблаговременное обнаружение опасности возникновения заторов на пересечениях транспортных потоков, вызванных ограниченной пропускной способностью этих участков. Наискорейшее обнаружение опасности затора даёт возможность соответствующим дорожным службам успеть подготовиться к этой проблеме и принять какие-либо меры для её предотвращения. Пропускная способность перекрёстка зависит от множества факторов, таких как состав движения, угол обзора, качество покрытия. Но основным параметром является интенсивность движения на встречном и перекрёстных направлениях.

Для измерения интенсивности движения на участке автодороги, современные технические средства позволяют определять время отделяющее два ближайших автомобиля. Тогда определение моментальной интенсивности движения на данном участке является математическим ожиданием этого времени умноженное на коэффициент, связанный с величиной интервала времени, за который определяется интенсивность движения. Тогда вычисляемые данным способом интенсивности составляют случайный процесс. 

Целью работы является разработка программного комплекса, позволяющего с заданной точностью определить, сменилась ли интенсивность движения, насколько она сменилась, и появилась ли в новой ситуации опасность возникновения затора. 

\section{Обзор современных методов обнаружения разладки в случайных процессах}

В современном мире методы теории принятия решений в условиях неопределённости используются довольно широко. Есть несколько классических книг, посвящённых теории принятия решений, нацеленных на решение конкретных задач общего интереса, возникающих при динамическом анализе (в режиме реального времени) статистических данных, получаемых, например, в финансовой инженерии, в теории обнаружения сигналов на фоне помех и т.д. В первую очередь стоит отметить такие монографии как [1] и [2]. 

Из русскоязычных авторов наибольший вклад в изучение этой теории осуществил Альберт Николаевич Ширяев. В частности, вопрос об оптимальной остановке Винеровского процесса был подробно рассмотрен в книге [3]. Для понимания которой необходимо также ознакомиться и с такими трудами как [4] и [5]. В данной работе автор рассматривает методы обнаружения нежелательных внедрений, основанного на Anomaly Detection Systems. 

Обычно, внедрение в сети происходит в неизвестный заранее момент времени $\theta$ и сопровождается изменением вероятностно-статистических свойств некоторых характеристик наблюдаемого процесса(например, количества отправленных и принятых сервером пакетов). Поэтому естественно возникает идея математически сформулировать задачу обнаружения атаки как задачу скорейшего обнаружения момента $\theta$ появления разладки в ходе наблюдаемого процесса. 

Основным аппаратом решения таких задач является последовательный анализ принятия решений. Такие задачи удобно формулировать как задачи об оптимальной остановке. Методы их решения в значительной мере опираются на современный аппарат теории случайных процессов, стохастического исчисления, теории мартингалов, нелинейной фильтрации и т.д.

Для начала рассмотрим ключевые статистики и тесты от наблюдаемых данных, на основании которых принимаются оптимальные решения. 

Предположим, что наблюдаемые данные описываются числовой последовательностью

$$
x_1, x_2, ..., x_n, ...
$$

являющейся результатом наблюдений над независимыми одинаково распределёнными случайными величинами 

$$
\xi_1, \xi_2, ..., \xi_n, ...
$$

Далее мы будем считать, что $\xi_k$~--- это одномерные случайные величины, и предполагать, что одномерные функции распределения $F_{\theta} = F_{\theta}(x) ( = P_{\theta} (\xi_n \leq x))$ имеют (при любых $n \geq 1$) плотности $f_\theta(x)$:
$$
dF_\theta(x) = f_\theta(x) dx
$$

Независимость и одинаковая распределённость означают, что плотность $p_\theta(x_1, x_2, ..., x_n)$ совместного распределения $F_\theta(x_1, x_2, ..., x_n) = P_\theta(\xi_1 \leq x_1, \xi_2 \leq x_2, ..., \xi_n \leq x_n)$ имеет следующий вид:
$$
p_\theta(x_1, x_2, ..., x_n) = f_\theta(x_1) f_\theta(x_2) ... f_\theta(x_n)
$$

Одной из ключевых статистик (для каждого $n \geq 1$) будет статистика 
$$
L_n = \frac{f_0(x_1) f_0(x_2) ... f_0(x_n)}{f_\infty(x_1) f_\infty(x_2) ... f_\infty(x_n)}
$$

Исключительная роль этих статистик $L_n$, $n \geq 1$, называемых отношениями правдоподобия, проявляется в задаче различения двух гипотез $H_0$ и $H_\infty$ (по $N$ наблюдениям) о том, какую именно плотность, $f_0(x)$ или $f_\infty(x)$, имеют наблюдаемые случайные величины $\xi_1, \xi_2, ..., \xi_N$. Решение этой задачи даётся так называемой леммой Неймана-Пирсона(см. [1, гл. 1] ширяев и [13,гл 3] стр. 18).

 
 
 
 
 
 
 
 
 
 


\subsection{Концептуальная модель компьютерной имитации транспортных потоков на двухполосных автомобильных дорогах}

\section{Описание }

Для описания движения автомобилей по автомобильным дорогам, городским улицам и автомагистралям разработаны многочисленные математические модели [5, 6, 10, 12, 13, 16, 18, 23, 34, 35, 41 и др.].

Разработка математической модели транспортного потока или математическое описание его характеристик является одним из важнейших и ответственных этапов при решении задач количественной и качественной оценки функционирования системы ВАДС, поскольку именно на этом этапе решается вопрос, насколько точно будут учтены те характеристики транспортного процесса и количественные связи между ними, которые необходимы для решения конкретных поставленных задач.

Строгого определения понятия сложной системы не существует. Основными отличительными признаками сложных систем являются:
\begin{itemize}
\item наличие большого количества взаимосвязанных и взаимодействующих элементов;
\item сложность функций, выполняемой системой и направленной на достижение заданной цели функционирования;
\item возможность разбиения системы на подсистемы, цели функци-онирования которых подчинены общей цели функционирования системы;
\item наличие управления, разветвленной информационной сети и интенсивных потоков информации;
\item наличие взаимодействия с внешней средой и функционирова-ние в условиях воздействия случайных факторов [7].
\end{itemize}

Дорожное движение обладает всеми перечисленными свойствами. Естественно рассматривать его в виде сложной системы и обозначать \glqq водитель – автомобиль – дорога – окружающая среда\grqq (ВАДС) [35].

Совершенствование электронно-вычислительной техники наряду с достижениями теории сложных систем привело к возникновению и успешному развитию нового направления в области исследования сложных систем~--- машинной имитации. По определению Т. Нейлора, машинная имитация~--- это численный метод проведения компьютерных экспериментов с математическими моделями, описывающими поведение сложной системы в течение продолжительных периодов времени [29]. Главным преимуществом метода машинной имитации является возможность проведения экспериментов не с реальной системой, а с ее математической моделью, реализованной на компьютере; при этом не существует ограничений по проведению экспериментов с целью изучения любых характеристик движения автомобиля [35].

Членом - корр. АН СССР Н.П. Бусленко и его школой предложена унифицированная абстрактная схема агрегата, которая позволяет единообразно описывать все элементы сложной системы (дискретные, непрерывные, детерминистические, стохастические) [7 и др.].

Общий принцип работы предлагаемой имитационной модели соответствует теории имитационного моделирования транспортных потоков, разработанной одним из авторов данной монографии [13]. Здесь для моделирования сложной системы ВАДС используется схема кусочно-непрерывного агрегата, с математической точки зрения представляющий собой условный марковский процесс с кусочно-непрерывными траекториями в пространстве переменной размерности. Конкретная реализация имитационной модели в виде микроописания отдельных блоков, а также полностью отлаженного программного комплекса, предназначенного для оценки степени опасности дорожного движения, пропускной способности, потерь времени, других характеристик транспортного потока на заданном участке дороги осуществлена другим автором [5] на базе языков программирования FORTRAN и Visual C++.

Агрегат характеризуется множествами моментов времени $Т$, состояний в каждый момент времени $Z$, входных $X$ и выходных $Y$ сигналов. Состояние агрегата в момент $t \in T$ обозначается $z(t) \in Z$, входные и выходные сигналы соответственно~--- $x(t) \in X$, и $y(t) \in Y$.

Состояние агрегата в момент $(t+0)$ обозначим $z(t+0)$. Предполагается, что из состояния $z(t)$ в состояние $z(t+0)$ агрегат приходит за малый интервал времени. Переход агрегата из состояния $z(t_1)$ в $z(t_2)$,  $t_2 > t_1$, определяется динамическими свойствами самого агрегата и входными сигналами.

Во множестве состояний $Z$ выделяется такое подмножество $Z^{(W)}$, что если в момент $t'$  $z(t')$ достигает $Z^{(W)}$, то агрегат скачкообразно изменяет свое состояние. Пусть эти изменения описываются оператором $W$:
$$
z(t'+0) = W[t', z(t')] \eqno(1)
$$
В случае воздействия входного сигнала $x_n$ поведение модели описывается оператором  $V$. Тогда состояние $z(t_n+0)$, где $t_n$~--- момент поступления в агрегат входного сигнала $x_n, t_n \in T$,

$$
z(t_n+0)=V[t_n, z(t_n), x(t_n)]\eqno(2)
$$

Если интервал $(t_n, t_{n+1})$ не содержит ни одного момента поступления сигналов, то для $t_n \in (t_n, t_{n+1}]$ состояние агрегата определяется оператором

$$
z(t) = U[t, t_n, z(t_n+0)]\eqno(3)
$$
Совокупность операторов $W$, $V$ и $U$ рассматривается как оператор переходов агрегата в новое состояние.

Во множестве состояний $Z$ выделяется подмножество $Z^{(Y)}$ $(Z^{(W)}$ и $Z^{(Y)}$ могут пересекаться) такое, что если $z(t*)$ достигает $Z^{(Y)}$, то выдается выходной сигнал, который определяется оператором выходов

$$
y=G[t*, z(t*)]\eqno(4)
$$

Упорядоченная совокупность рассмотренных множеств $T$, $X$, $Z$, $Z^{(W)}$, $Z^{(Y)}$ ,$Y$ и случайных операторов $W$, $V$, $U$, $G$ полностью задает агрегат как динамическую систему.
Определенного рода упорядоченная совокупность конечного числа агрегатов называется агрегативной системой [7].

Агрегативные системы в качестве математической модели транспортных потоков в России использовались, начиная с 1980-х годов, в исследованиях В.В. Сильянова, В.М. Еремина, В.Г. Крбашяна, Р.С. Картанбаева, М.С. Талаева, А.И. Должикова, О.И. Тонконоженкова, С.П. Крысина и других. В этих исследованиях рассмотрены различные проблемы проектирования автомобильных дорог и безопасности дорожного движения.

Объектом данного исследования является дорожное движение на различных фрагментах сети двухполосных автомобильных дорог. Вся сеть дорог представлена в виде графа, который состоит из узлов и ребер. Узлами являются перекрестки (примыкания) вместе с подходами к нему. Длина каждого из подходов определяется расстоянием влияния перекрестка и составляет 150~--- 200 м. Ребрами являются двухполосные перегоны. Математической моделью каждого элемента графа, т.е. узла или ребра является кусочно-непрерывный агрегат. Очевидно, что если входные и выходные контакты соответствующих агрегатов соединить каналами связи, то можно получить агрегативную систему, которая соответствует любому заданному подмножеству дорожной сети. При этом количество моделируемых элементов дорожной сети ограничивается только техническими параметрами применяемой вычислительной техники (оперативная память, быстродействие и т.д.). Таким образом, моделируемая система ВАДС состоит из двух типов агрегатов: <<Узел>> и <<Ребро>>.

Ниже, если особо не будет оговорено, будем рассматривать наиболее общий случай кусочно-непрерывного агрегата <<Узел>>~--- перекресток с четырьмя подходами (рис. 2.1).

Рис. 2.1. Схема моделируемого нерегулируемого перекрестка

Продольные оси всех подходов к перекрестку пересекаются в точке $О$. Углы между ними $\alpha_1$, $\alpha_2$, $\alpha_3$, $\alpha_4$, могут принимать любые значения, в соответствии с конфигурацией конкретного перекрестка. Обычно $\alpha_1 = \alpha_3$, и $\alpha_2 = \alpha_4$.
Выберем неподвижную прямоугольную систему координат $ХОУ$ с центром в точке $О$. Пусть координатная ось ОХ направлена вдоль продольной оси одного из подходов к перекрестку (1-й подход). Весь перекресток разбит на 9 подсистем:

\begin{itemize}
\item $L = 1, 3, 5, 7$ – полосы движения, подходящие к перекрестку;
\item $L = 2, 4, 6, 8$ - полосы движения, отходящие от перекрестка;
\item $L = 9$ – перекресток.
\end{itemize}
Каждая из полос движения $L=1 \div 8$ в свою очередь может быть разбита на несколько участков, которые отличаются друг от друга геометрическими параметрами и/или средствами организации движения (элементарный участок). Перекресток ($L=9$) состоит из 12 элементар-ных участков (рис. 2.2). Под элементарным участком на перекрестке понимается участок полосы движения, соединяющий конец каждого из четырех нечетных полос движения с началом каждой из четных полос движения (кроме сопряженной четной полосы движения).

Рис. 2.2. Нумерация элементарных участков на перекрестке

% Множество моментов времени $Т$ агрегата задается вектором $t_{i,j}$, где $i$ - номер автомобиля (способ нумерации автомобилей описан ниже); $j$ – номер ситуации, при наступлении которого состояние агрегата скачкообразно меняется. Размерность вектора Т во многом определяет скорость проведения имитационного эксперимента. В модели предусмотрено 30 ситуаций, а также одновременное присутствие в системе 1500 автомобилей.

Состояние агрегата Z описывается постоянными параметрами системы и переменными координатами. К постоянным относятся:
\begin{enumerate}
\item Параметры каждого элементарного участка дороги:

\begin{itemize}

\item длина элементарного участка дороги по продольной оси;
\item расстояние видимости до встречного автомобиля в плане;
\item расстояние видимости до встречного автомобиля в продольном профиле;
\item коэффициент сцепления шины с поверхностью дороги;
\item продольный уклон участка дороги;
\item поперечный уклон участка дороги;
\item ширина проезжей части;
\item ширина обочины;
\item радиус кривизны в плане;
\item радиус вертикальной кривой в плане;
\item координата начала продольной оси участка дороги по оси OX;
\item координата начала продольной оси участка дороги по оси OY;
\item направление полосы движения;
\item параметры уравнения продольной оси участка дороги в аналитическом виде;
\item ограничение скорости движения легковых автомобилей;
\item ограничение скорости движения средних грузовых автомобилей;
\item ограничение скорости движения тяжелых грузовых автомоби-лей;
\item ограничение скорости движения автобусов;
\item ровность дорожного покрытия;
\item ровность дорожного покрытия;
\item обгон запрещен;
\item обгон грузовым автомобилям запрещен;
\item коэффициент сопротивления качению при скорости движения автомобиля 20 км/ч;
\item угловой коэффициент скорости движения для определения коэффициента сопротивления качению.
\end{itemize}

\item Объекты окружающей среды, ограничивающие видимость.

\item Транспортные потоки, задающиеся следующими параметрами:
\begin{itemize}
\item интенсивность в начале полосы движения;
\item распределение интенсивности по направлениям движения (направо, прямо, налево);
\item состав потока на каждой полосе движения. Транспортный поток в модели состоит из свыше 20 наиболее типичных отечественных и зарубежных марок автомобилей, которые сгруппированы по 4 типам: легковые, средние грузовые, тяжелые грузовые, автобусы.
\end{itemize}
\item Тягово–динамические и технико–эксплуатационные характеристики автомобилей всех марок:
\begin{itemize}
\item передаточное число  k–й передачи коробки перемен передач (КПП) автомобиля;
\item передаточные числа главной передачи;
\item передаточное число делителя КПП;
\item габаритная длина автомобиля;
\item габаритная ширина автомобиля;
\item габаритная высота автомобиля;
\item база;
\item колея;
\item передний свес;
\item расстояние от переднего бампера до глаз водителя по продоль-ной оси;
\item коэффициент обтекаемости;
\item собственная масса;
\item нагрузка на переднюю ось;
\item нагрузка на заднюю ось;
\item полная масса;
\item нагрузка на переднюю ось при полной массе;
\item нагрузка на заднюю ось при полной массе;
\item параметры для определения коэффициента учета вращаю-шихся масс;
\item максимальный крутящий момент двигателя;
\item угловая скорость коленчатого вала, соответствующая макси-мальному крутящему моменту;
\item максимальная мощность двигателя;
\item угловая скорость коленчатого вала, соответствующая макси-мальной мощности двигателя;
\item время срабатывания тормозного привода;
\item время запаздывания тормозного привода и времени нарастания замедления (при торможении);
\item передаточное число рулевого механизма;
\item коэффициент полезного действия трансмиссии;
\item коэффициент коррекции двигателя;
\item свободный радиус колеса автомобиля;
\item динамический радиус колеса;
\item радиус поворота по следу внешнего переднего колеса;
\item радиус поворота по следу внешней, габаритной точки.
\end{itemize}

\item Закон распределения коэффициента использования грузоподъемности (загрузки) автомобилей.
\end{enumerate}

К переменным координатам состояния агрегата относятся:

\begin{enumerate}
\item Количество автомобилей на полосе движения и на перекрестке в текущий момент времени.
\item Координаты всех моделируемых автомобилей, зависящие от следующих параметров:

\begin{itemize}
\item пройденный путь;
\item координата автомобиля по продольной оси дороги;
\item расстояние автомобиля от продольной оси в поперечном направлении (принимает положительное значение, если автомобиль расположен левее от продольной оси и отрицательные значения, если правее);
\item скорость движения;
\item тангенциальное ускорение;
\item нормальное ускорение;
\item время, прошедшее с момента формирования автомобиля;
\item курсовой угол автомобиля в системе координат $ХОУ$ [рис. 2.1];
\item угол наклона;
\item крен автомобиля;
\item угол поворота управляемых колес;
\item угловая скорость поворота рулевого колеса;
\item угловая скорость поворота управляемых колес;
\item координаты автомобиля, соответственно по оси $OX$ и $OY$;
\item скорость движения, соответственно по оси $OX$ и $OY$;
\item ускорение автомобиля, соответственно по оси $OX$ и $OY$;
\item угол поворота управляемых колес, при котором водитель пре-кращает поворот рулевого колеса на 1-ом и 7-ом этапе маневра обгона (или на 1-ом этапе перестроения);
\item то же самое, но на 4-ом и 10-ом этапе обгона [рис. 2.11];
\item угловая скорость поворота управляемых колес при обгоне;
\item длина прямолинейного участка на 3-м и 9-ом этапе обгона;
\item тип водителя по скорости свободного движения;
\item время реакции водителя на изменение режима движения лидера;
\item время реакции водителя при появлении помехи на перекрестке;
\item время, оставшееся до истечения времени реакции водителя на изменение режима движения лидера;
\item то же самое, но после наступления другого события;
\item коэффициент использования грузоподъемности;
\item угловая скорость поворота управляемых колес на 3-м этапе поворота;

Под координатами автомобиля понимаются координаты середины задней оси автомобиля.

\item номер элементарного участка дороги или перекрестка, на котором находится автомобиль в данный момент времени;
\item марка автомобиля;
\item тип автомобиля;
\item состояния в процессе обгона (обгоняющий, обычный, обгоняемый);
\item количество обгоняемых автомобилей, находящихся впереди;
\item номер автомобиля в пачке;
\item тип передачи КПП;
\item направление дальнейшего движения после перекрестка (налево, прямо, направо);
\item состояние сигналов поворота (включен сигнал левого поворота, выключен, включен сигнал правого поворота);
\item состояние стоп–сигналов (включены, выключены);
\item следование за лидером [(см. п. 2.3, ситуации 1-4)];
\item номер ближайшего встречного автомобиля;
\item номер ближайшего встречного автомобиля, который находится в зоне видимости;
\item номер автомобиля – лидера, за которым следует данный автомобиль;
\item номер этапа в процессе поворота (всего 3 этапа);
\item номер этапа в процессе обгона (всего 11 этапов);
\item индекс приоритетов (помеха справа, помеха слева, помехи отсутствуют);
\item тип водителя по реакции на ограничение скорости движения (нарушает правила дорожного движения (ПДД) или не нарушает ПДД);
\item тип водителя по реакции на дорожный знак <<Обгон запрещен>> (нарушает требования дорожного знака или не нарушает);
\item тип водителя по дистанции следования (<<осторожный>>, <<обычный>>, <<рискованный>>);
\item тип водителя по оценке обстановки около перекрестка (изменение скорости движения заранее до безопасной величины, изменение скорости движения только при появлении в зоне видимости помехи справа);
\item тип водителя по включению сигналов поворота (включает или не включает);
\item номер ситуации, после наступления которого, водитель принял решение изменить режим движения;
\item положение, куда попадет автомобиль после перекрестка
\item наличие или отсутсвие встречного автомобиля на своей полосе движения.
\end{itemize}

\item Приоритет проезда через перекресток.
\end{enumerate}

В зависимости от необходимого объема выходной информации задается время моделирования $t_{мод}$, которое в дальнейшем по мере работы модели уменьшается до нуля. Вместо истечения времени моделирования предусмотрены также другие правила остановки модели: прохождение через заданный участок дороги определенного количества автомобилей; накопление определенного количества статистической информации и т.д.

С начала работы модели задается период разогрева $t_{раз}$. В течение времени $t_{раз}$ система выходные сигналы не выдает. Необходимость ввода времени разогрева вызвано тем, что в момент начала работы модели на моделируемом фрагменте дорожной сети автомобили отсутствуют. Для того чтобы транспортные потоки принимали те параметры, которые были введены в качестве исходных данных, необходим некоторый промежуток времени, за которое самый медленный автомобиль проедет через самый длинный (продолжительный по времени) маршрут на рассматриваемой дорожной сети. Только после этого можно считать, что модель <<разогрелась>>. 

 


\section{Концептуальная постановка задачи}

Каждый элемент улично-дорожной сети (УДС) характеризуется своей пропускной способностью. Пропускная способность отдельных участков УДС определяется пропускной способностью узкого места. Узкими местами являются пересечения и примыкания автомобильных дорог. Одной из основных задач интеллектуальных транспортных систем (ИТС) является предсказание появления заторов на участках УДС, которые образуются в случае превышения пропускной способности элементов УДС (в нашем случае~--- пересечение автомобильных дорог). Чем раньше и надёжнее мы сможем оценить интенсивность движения на подходе к узкому месту, тем эффективнее будут действия ИТС по предотвращению заторов. Таким образом основной задачей данной работы является определение достоверной интенсивности движения на всех возможных подходах к узкому месту с возможностью минимизации времени фиксирования разладки. 

Поскольку формирование транспортного потока на подходах к узкому месту являются случайными процессами, то в качестве математического аппарата используются вероятностно статистические методы в теории принятия решений.

Поскольку число пересечений автомобильных дорог различных типов достаточно велико, то в качестве узкого места в данной работе было выбрано нерегулируемое, равнозначное пересечение двухполосных дорог в одном уровне. Такого типа пересечения являются наиболее сложными в описании их функционирования по пропуску транспортных потоков. Перейдём к формализации постановки задачи.

\section{Формальная постановка задачи}

Интенсивностью движения по одной полосе называется число автомобилей, проходящих в единицу времени через заданный створ полосы движения. Пропускной способностью полосы движения называется максимальное число автомобилей, проходящих в единицу времени через конечный створ полосы движения. Пропускная способность полосы движения зависит от многих факторов: геометрических характеристик полосы, состав её покрытия, видимость, состав транспортного потока и так далее.

Пропускную способность пересения определить не так то просто. 
%рисунок
На рисунке \ref{pict1} представлены возможные маршруты движения транспортных потоков на нерегулируемом равнозначном двухполосном пересечении двух дорог. Мы видим, что всего таких маршрутов 12. При этом каждый из них в принципе может быть независим от всех остальных, поэтому не совсем понятно что в таком случае считать пропускной способностью пересечения. Пропускная способность определяется по работам [1,2,3,...]
%работы
Исходя из этих работ пропускной способностью пересечения является многомерная функция, зависящая от 11 переменных. Данная функция строится по результатам компьютерных экспериментов с имитационными моделями транспортных потоков, движущихся по пересечению. Компьютерные эксперименты для определения указанной функции заключаются в следующем. Фиксируется значение интенсивности движения в одиннадцати маршрутах, а для двенадцатого маршрута проводятся серии экспериментов для увеличивающихся щначений интенсивности движения по нему до тех пор, пока не будет достигнута пропускная способность данного маршрута. Для получения указанной функции требуется проведение сотен тысяч и боее компьютерных экспериментов с соответствующем семейством имитационных моделей. Такие эксперименты были проведены для рассматриваемого в данной рабооте пересечения и их результаты используются в работе в качестве исходных данных. 

Предполагается, что фрагмент рассматриваемой ИТС состоит из пунктов наблюдения за четырьмя интенсивностями движения на подходе к пересечению, которые могут находиться на значительном расстоянии от пересечения (2--4 километра и более). Задача ИТС состоит в определении точного значения интенсивности движения на подходах и проведении прогноза возможности образования заторов на пересечении. 

Рассмотрим в отдельности один подход к пересечению. Пусть в заданном его створе происходит сбор информации о проходящих автомобилях. 

Предполагается, что в пункте наблюдения (створ полосы движения) случайным образом появляются автомобили. Случайные интервалы между появляением двух соседних автомобилей могут быть распределены по разным законам (Пуассона, Эрланга, Гаусса). Пункт наблюдения должен проводить оценку интенсивности движения за конкретный промежуток времени(5 минут, 10, и т.д.) или после появления определённого числа автомобилей (50, 100 и т.д.). Значение интенсивности движения есть случайная величина. Задача ИТС заключается в определении времени наступления и величины разладки с заданной точностью. Эта задача решается в пункте [8]
%решение задачи разладки
Собрав такого рода информацию на всех подходах к пересечению определяется вероятность появления заторов, т.е. превышения пропускной способности пересечения, о чём говорилось выше.

Желательно, чтобы система так же давала предупреждения о значениях интенсивности близкой к пропускной способности.

\section{Математическая модель}

\subsection{Математическая постановка задачи}

Пусть имеется числовая последовательность $x = (x_1, x_2, ...)$ с $x_i \in \mathbb{R}$ представляющая собой Вальдовский процесс. \textbf{ОПИСАТЬ ВАЛЬДОВСКИЙ ПРОЦЕСС}. Наименьшую $\sigma$-алгебру в $\mathbb{R}$, порождённую множествами вида
\begin{displaymath}
  \{ x: x_1 \in I_1, ... x_n \in I_n \}, n \ge 1,
\end{displaymath}
где $I_k$~--- борелевские множества на $\mathbb{R}$, обозначим через $\mathcal{B}$. 

Будем считать, что на $(\mathbb{R}, \mathcal{B})$ заданы две вероятностные меры $P_0$ и $P_\infty$. Через $f_\theta$ будем обозначать плотность распределения 

\begin{displaymath}
  P_\theta(\xi \in B) = \int_B f_\theta(x)\mu(dx), \theta = 0, \infty,
\end{displaymath}

Предположим, что шаг за шагом мы получаем данные $x_1, x_2, ...$, являющиеся наблюдениями над случайными величинами $\xi_1, \xi_2, ...$. Мы хотим различить две гипотезы $H_0$ и $H_\infty$~--- о том, какое действует распределение, $P_0$ или $P_\infty$,~--- используя последовательные тесты, определяемые следующим образом. 

Каждый последовательный тест $\delta$ определяется парой $(\tau, \phi)$, где 

\begin{enumerate}
  \item $\tau = \tau(x)$~--- марковский момент (или момент остановки) относительно потока $\{\mathcal{F}_n, n \ge 1\}$, где $\mathcal{F}_n = \sigma(x: x_1, x_2, ..., x_n)$~--- $\sigma$-алгебра, порождённая наблюдениями $x_1, x_2, ..., x_n$, т.е. $\tau = \tau(x)$~--- момент остановки, принимающий значения $0, 1, ..., \infty$ и такой, что 
  \begin{displaymath}
    \{x: \tau(x) \le k\} \in \mathcal{F}_k
  \end{displaymath}
  при каждом $k \in \{0, 1, ...\}$
  
  \item $\phi = \phi(x)$~--- $\mathcal{F}_\tau$-измеримая функция со значениями в $[0, 1]$, где $\mathcal{F}_\tau = \sigma(x: x_1, x_2, ..., x_\tau)$~--- $\sigma$-алгебра, порождённая величинами $x_1, x_2, ..., x_\tau$.
\end{enumerate}

Момент $\tau$ интерпретируется как момент прекращения наблюдений с последующим принятием решения $\phi = \phi(x)$, интерпретируемого как вероятность принятия гипотезы $H_0$, когда наблюдениями являются $x_1, x_2, ..., x_\tau$.

ОПИСАТЬ МНОГОКРАТНЫЕ МОМЕНТЫ ОСТАНОВКИ. МОМЕНТ РАЗЛАДКИ.

Для решения задачи обратимся к последовательному тесту Вальда.

\textbf{ТЕСТ ВАЛЬДА}

\subsection{Приложение к конкретной задаче}

Броуновское движение. 

Фиксированное значение среднеквадратичного отклонения.

Обнаружение изменения математического ожидания. 

Оптимальное теоретическое значение $\mu_\infty$.

Определение значения математического ожидания после разладки.

\section{Программная реализация}

Результаты работы.




% 
% \begin{figure}[!h]
%   \includegraphics[width=150mm]{task.png}
%   \hfill\hfill
%   \caption{\it}
% \end{figure}
% 
% Построить график зависимости коэффициента $K$ от расстояния $R_0$. При каких значениях $R_0$ классическую формулу закона притяжения можно считать выполняющейся с точностью не хуже $30, 20, 10, 5, 1, 0.5 \%$?
% 
% Дополнительные вопросы:
% \begin{enumerate}
%   \item Как изменятся результаты, если параллелепипед развернуть на $90^{\circ}$ вокруг оси y, так что он станет вытянут вдоль оси $z$?
%   \item Как изменятся результаты, если масса $m$ также будет представлять из себя однородный параллелепипед с размерами $a \times b \times c$?
%   \item Насколько существенно изменятся результаты, если вместо параллелепипеда рассмотреть однородный шар радиуса $R$?
% \end{enumerate}
% 
% 
% \section{Решение}
% 
% Для нахождения коэффициента неточечности масс нужно найти значение силы тяжести по классическому закону тяготения Ньютона. После этого найдём силу гравитации для точечной части параллелепипеда: $dF = G\frac{m dM}{R^2}dV$. Проинтегрировав, получим: $F = G \cdot m \cdot M \int_{dV} \frac{dx dy dz}{(R_0 + z)^2 + y^2 + x^2}$ 
% 
% Методом Монте-Карло найдём интеграл и поделив второе значение на первое получим коэффициент неточечности масс. После чего повторим эти расчёты, увеличивая расстояние между телами. Получим график 1.
% 
% \begin{figure}[!h]
%    \includegraphics[width=150mm]{first.png}
%      \hfill\hfill
%   \caption{\it}
% \end{figure} 
% 
% Из расчётов можно сделать вывод, что если параллелепипед развернуть на $90^{\circ}$ вокруг оси y, так что он станет вытянут вдоль оси $z$, то коэффициент $K$ уменьшится, но незначительно.
% 
% Вторая часть задания состоит в нахождении коэффициента неточечности масс для двух неточечных тел (двух параллелепипедов). Для этого найдём силу гравитации для точечных частей параллелепипедов. $ dF = \int \frac{dm dM}{R^2} dV1 dV2 $. Проинтегрировав, получим: 
% \begin{math}
%   F = G \cdot m \cdot M \int_{dV} \int_{dV_1} \int_{dV_2} \frac{dx_1 dy_1 dz_1 dx_2 dy_2 dz_2}{(x_1 - x_2 + R_0)^2 + (y_1 - y_2)^2 + (z_1 - z_2)}
% \end{math}
% 
% Методом Монте-Карло найдём интеграл и поделив второе значение на первое получим коэффициент неточечности масс. После чего повторим эти расчёты, увеличивая расстояние между телами. Получим график 2.
% 
% \begin{figure}[!h]
%    \includegraphics[width=150mm]{second.png}
%      \hfill\hfill
%   \caption{\it}
% \end{figure} 
% 
% Третья часть задания состоит в нахождении коэффициента неточечности масс для шара радиуса R. Для этого найдём силу гравитации для точечной части шара. $dF = G \cdot m \int_{dV}\frac{m dM}{R^2}dV$. Для интегрирования полученного выражения, перейдём к сферическим координатам. После перехода и интегрирования получим:
% \begin{math}
%   F = G \cdot m \cdot M \int_{dV} \frac{d \phi d \theta dr}{r^2 sin \phi \sqrt{(r\cos\phi\sin\theta)^2 + (r\sin\phi\sin\theta)^2 + (R_0 - r\cos\phi)^2}}
% \end{math}
% 
% Методом Монте-Карло найдём интеграл и поделив второе значение на первое получим коэффициент неточечности масс. После чего повторим эти расчёты, увеличивая расстояние между телами. Получим график 3.
% 
% \begin{figure}[!h]
%    \includegraphics[width=150mm]{third.png}
%      \hfill\hfill
%   \caption{\it}
% \end{figure} 
% 
% Из этих трёх графиков видна обратная зависимость коэффициента неточечности масс от растояния. Т.е. чем больше расстояние, тем меньше коэффициент. Поэтому на больших расстояниях можно пользоваться Ньютоновской формулой нахождения силы гравитации.
% 
% 

\end{document}



Общий план:

Введение, в котором описывается актуальность темы, примерная постановка проблемы и примерное описание способа решения поставленной задачи.

В литера

Далее должна идти постановка задачи, которую мне продиктовал Валерий Михайлович. После постановки задачи концептуальной и формальной, идёт постановка математическая.

Далее


Должно быть теоретическое описание общей модели определения интенсивности, и определения пропускной способности.

В теоретической части обзора литературы, нужно рассмотреть:
Текст из введения 
Предлагаемые статистики для обнаружения разладки.
Возможные процессы, показывающие разладку. (Tn, Ym...) Вставить графики.
Сравнение методов Неймана-Пирсона и Вальда.




Что есть:
Введение
Описание модели перекрёстка

Математическая постановка задачи о разладке


